\documentclass{article}

\usepackage{color}
\usepackage{amsmath}
\usepackage{amsthm}
\usepackage{amssymb}
\usepackage{latexsym}
\usepackage{amsfonts}
%\usepackage{times}
\usepackage{url}
%\usepackage{bibspacing}
%\setlength{\bibspacing}{\baselineskip}
\usepackage{hyperref}
\usepackage{xspace}
\usepackage{graphicx}
\usepackage{alltt}

 \definecolor{grey}{rgb}{0.9,0.9,0.9}



\newtheorem{thm}{Theorem}[section]
\newtheorem{lem}[thm]{Lemma}
\newtheorem{cor}[thm]{Corollary}
\newtheorem{propo}[thm]{Proposition}
\newtheorem{fct}[thm]{Fact}
\newtheorem{defn}[thm]{Definition}
\newtheorem{exmp}[thm]{Example}
\newtheorem{assm}[thm]{Assumption}
\newtheorem{clm}[thm]{Claim}
\newtheorem{techclm}[thm]{Technical Claim}
\newtheorem{rem}[thm]{Remark}
\newtheorem{cons}[thm]{Construction}

\newenvironment{theorem}{\begin{thm}\begin{rm}}%
{\end{rm}\end{thm}}
\newenvironment{lemma}{\begin{lem}\begin{rm}}%
{\end{rm}\end{lem}}
\newenvironment{corollary}{\begin{cor}\begin{rm}}%
{\end{rm}\end{cor}}
\newenvironment{proposition}{\begin{propo}\begin{rm}}%
{\end{rm}\end{propo}}
\newenvironment{fact}{\begin{fct}\begin{rm}}%
{\end{rm}\end{fct}}
\newenvironment{definition}{\begin{defn}\begin{em}}%
{\end{em}\end{defn}}
\newenvironment{example}{\begin{exmp}\begin{em}}%
{\end{em}\end{exmp}}
\newenvironment{assumption}{\begin{assm}\begin{em}}%
{\end{em}\end{assm}}
\newenvironment{claim}{\begin{clm}\begin{rm}}%
{\end{rm}\end{clm}}
\newenvironment{techclaim}{\begin{techclm}\begin{rm}}%
{\end{rm}\end{techclm}}
\newenvironment{remark}{\begin{rem}\begin{em}}%
{\end{em}\end{rem}}
\newenvironment{construction}{\begin{cons}\begin{em}}%
{\end{em}\end{cons}}



\newcommand{\Succ}{\mathsf{Succ}}
\newcommand{\Adv}{\mathbf{Adv}}
 \newcommand{\Exp}{\mathbf{Exp}}
\begin{document}

\section{Motivation}
Protect information contained in statistical database, i.e, if a database is a representative sample of an underlying, the goal of a privacy preserving statistical database is to enable the user to learn properties of the population as a whole , while protecting the privacy of the individuals in the sample.\\
What we want to achieve is something close to semantic security. Dalenius in 1977 articulated a desired property for database: nothing abouth an individual should be learnable from a database that cannot be learned without access to the database. This kind of privacy cannot be achieved: the obstacle is \textit{auxiliary information}. Here an example:\\
suppose one's exact height were considered sensitive information and revealing it were a privacy breach. Assume that the database yields the average heights of women of different nationalities. AN adversary who access the database and the auxiliary information ``Terry Gross is two inches shorter than the average lithuanian woman'' learns Terry Gross height while anyone learning only the auxiliary information without access to the database ,learns relatively little.\\
There are to remarks to do wrt the impossibility results:\\
\begin{itemize}
\item it applies regardless of whether or not Terry Gross is in the database
\item Dalenius goal cannot be achieved while semantic security for crypto yes.
\end{itemize}

The first of these
leads naturally to a new approach to formulating privacy goals: the risk to one’s
privacy, or in general, any type of risk, such as the risk of being denied automobile
insurance, should not substantially increase as a result of participating in a
statistical database. This is captured by \textit{differential privacy}

\section{Differential Privacy: the setting}
Given the impossibility result just showed we need to relax the definition of privacy: we move from absolute guarantees about disclosures to relative ones; any given disclosure will be, within  a small multiplicative factor.

\begin{definition}
A randomized algrotihm A gives $\epsilon$-differential privacy if for all
data sets $D_1$ and $D_2$ differing on at most one element, and all S $\subseteq$ Range(A),
\begin{equation}
Pr[A(D_1) \in S] \leq exp(\epsilon) \times Pr[A(D_2) \in S]
\end{equation}
\end{definition}

A mechanism K satisfying this definition addresses concerns that any participant
might have about the leakage of her personal information x: even if the participant
removed her data from the data set, no outputs (and thus consequences
of outputs) would become significantly more or less likely. For example, if the
database were to be consulted by an insurance provider before deciding whether
or not to insure Terry Gross, then the presence or absence of Terry Gross in the
database will not significantly affect her chance of receiving coverage.


\section{Achieving DP}




%\bibliographystyle{plain}
%\bibliography{crypto}
\end{document}